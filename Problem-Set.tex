% Options for packages loaded elsewhere
\PassOptionsToPackage{unicode}{hyperref}
\PassOptionsToPackage{hyphens}{url}
%
\documentclass[
]{article}
\usepackage{amsmath,amssymb}
\usepackage{lmodern}
\usepackage{iftex}
\ifPDFTeX
  \usepackage[T1]{fontenc}
  \usepackage[utf8]{inputenc}
  \usepackage{textcomp} % provide euro and other symbols
\else % if luatex or xetex
  \usepackage{unicode-math}
  \defaultfontfeatures{Scale=MatchLowercase}
  \defaultfontfeatures[\rmfamily]{Ligatures=TeX,Scale=1}
\fi
% Use upquote if available, for straight quotes in verbatim environments
\IfFileExists{upquote.sty}{\usepackage{upquote}}{}
\IfFileExists{microtype.sty}{% use microtype if available
  \usepackage[]{microtype}
  \UseMicrotypeSet[protrusion]{basicmath} % disable protrusion for tt fonts
}{}
\makeatletter
\@ifundefined{KOMAClassName}{% if non-KOMA class
  \IfFileExists{parskip.sty}{%
    \usepackage{parskip}
  }{% else
    \setlength{\parindent}{0pt}
    \setlength{\parskip}{6pt plus 2pt minus 1pt}}
}{% if KOMA class
  \KOMAoptions{parskip=half}}
\makeatother
\usepackage{xcolor}
\IfFileExists{xurl.sty}{\usepackage{xurl}}{} % add URL line breaks if available
\IfFileExists{bookmark.sty}{\usepackage{bookmark}}{\usepackage{hyperref}}
\hypersetup{
  pdftitle={Experimental Problem Set},
  pdfauthor={Brock Wilson},
  hidelinks,
  pdfcreator={LaTeX via pandoc}}
\urlstyle{same} % disable monospaced font for URLs
\usepackage[margin=1in]{geometry}
\usepackage{color}
\usepackage{fancyvrb}
\newcommand{\VerbBar}{|}
\newcommand{\VERB}{\Verb[commandchars=\\\{\}]}
\DefineVerbatimEnvironment{Highlighting}{Verbatim}{commandchars=\\\{\}}
% Add ',fontsize=\small' for more characters per line
\usepackage{framed}
\definecolor{shadecolor}{RGB}{248,248,248}
\newenvironment{Shaded}{\begin{snugshade}}{\end{snugshade}}
\newcommand{\AlertTok}[1]{\textcolor[rgb]{0.94,0.16,0.16}{#1}}
\newcommand{\AnnotationTok}[1]{\textcolor[rgb]{0.56,0.35,0.01}{\textbf{\textit{#1}}}}
\newcommand{\AttributeTok}[1]{\textcolor[rgb]{0.77,0.63,0.00}{#1}}
\newcommand{\BaseNTok}[1]{\textcolor[rgb]{0.00,0.00,0.81}{#1}}
\newcommand{\BuiltInTok}[1]{#1}
\newcommand{\CharTok}[1]{\textcolor[rgb]{0.31,0.60,0.02}{#1}}
\newcommand{\CommentTok}[1]{\textcolor[rgb]{0.56,0.35,0.01}{\textit{#1}}}
\newcommand{\CommentVarTok}[1]{\textcolor[rgb]{0.56,0.35,0.01}{\textbf{\textit{#1}}}}
\newcommand{\ConstantTok}[1]{\textcolor[rgb]{0.00,0.00,0.00}{#1}}
\newcommand{\ControlFlowTok}[1]{\textcolor[rgb]{0.13,0.29,0.53}{\textbf{#1}}}
\newcommand{\DataTypeTok}[1]{\textcolor[rgb]{0.13,0.29,0.53}{#1}}
\newcommand{\DecValTok}[1]{\textcolor[rgb]{0.00,0.00,0.81}{#1}}
\newcommand{\DocumentationTok}[1]{\textcolor[rgb]{0.56,0.35,0.01}{\textbf{\textit{#1}}}}
\newcommand{\ErrorTok}[1]{\textcolor[rgb]{0.64,0.00,0.00}{\textbf{#1}}}
\newcommand{\ExtensionTok}[1]{#1}
\newcommand{\FloatTok}[1]{\textcolor[rgb]{0.00,0.00,0.81}{#1}}
\newcommand{\FunctionTok}[1]{\textcolor[rgb]{0.00,0.00,0.00}{#1}}
\newcommand{\ImportTok}[1]{#1}
\newcommand{\InformationTok}[1]{\textcolor[rgb]{0.56,0.35,0.01}{\textbf{\textit{#1}}}}
\newcommand{\KeywordTok}[1]{\textcolor[rgb]{0.13,0.29,0.53}{\textbf{#1}}}
\newcommand{\NormalTok}[1]{#1}
\newcommand{\OperatorTok}[1]{\textcolor[rgb]{0.81,0.36,0.00}{\textbf{#1}}}
\newcommand{\OtherTok}[1]{\textcolor[rgb]{0.56,0.35,0.01}{#1}}
\newcommand{\PreprocessorTok}[1]{\textcolor[rgb]{0.56,0.35,0.01}{\textit{#1}}}
\newcommand{\RegionMarkerTok}[1]{#1}
\newcommand{\SpecialCharTok}[1]{\textcolor[rgb]{0.00,0.00,0.00}{#1}}
\newcommand{\SpecialStringTok}[1]{\textcolor[rgb]{0.31,0.60,0.02}{#1}}
\newcommand{\StringTok}[1]{\textcolor[rgb]{0.31,0.60,0.02}{#1}}
\newcommand{\VariableTok}[1]{\textcolor[rgb]{0.00,0.00,0.00}{#1}}
\newcommand{\VerbatimStringTok}[1]{\textcolor[rgb]{0.31,0.60,0.02}{#1}}
\newcommand{\WarningTok}[1]{\textcolor[rgb]{0.56,0.35,0.01}{\textbf{\textit{#1}}}}
\usepackage{graphicx}
\makeatletter
\def\maxwidth{\ifdim\Gin@nat@width>\linewidth\linewidth\else\Gin@nat@width\fi}
\def\maxheight{\ifdim\Gin@nat@height>\textheight\textheight\else\Gin@nat@height\fi}
\makeatother
% Scale images if necessary, so that they will not overflow the page
% margins by default, and it is still possible to overwrite the defaults
% using explicit options in \includegraphics[width, height, ...]{}
\setkeys{Gin}{width=\maxwidth,height=\maxheight,keepaspectratio}
% Set default figure placement to htbp
\makeatletter
\def\fps@figure{htbp}
\makeatother
\setlength{\emergencystretch}{3em} % prevent overfull lines
\providecommand{\tightlist}{%
  \setlength{\itemsep}{0pt}\setlength{\parskip}{0pt}}
\setcounter{secnumdepth}{-\maxdimen} % remove section numbering
\ifLuaTeX
  \usepackage{selnolig}  % disable illegal ligatures
\fi

\title{Experimental Problem Set}
\author{Brock Wilson}
\date{4/10/2021}

\begin{document}
\maketitle

\hypertarget{problem-1}{%
\section{Problem 1}\label{problem-1}}

In class, we discussed the Roy (1951) model of selection based on
comparative advantage. In this problem, we will simulate a slight
extension of the Roy Model to better understand non-compliance and local
average treatment effects. Specifically, we will assume people make
participation decisions entirely based on their earnings with or without
the training less any costs of the training. The extension is accounting
for the fact that different choices might have different costs.

For this exercise, assume we are evaluating the impact of earning a
certificate from a community college on a worker's earnings. If someone
is in the treatment group, they receive the training for free. But
someone in the control group can pay to enroll in the program on their
own for 1,000. Simulate a sample of 10,000 observations from the
following data generating process:

\[
Y_0 \sim N(20000, 7000^2) \\
Y_1 \sim N(21500, 8000^2)
\]

\begin{Shaded}
\begin{Highlighting}[]
\NormalTok{size }\OtherTok{=} \DecValTok{10000}

\NormalTok{y\_0 }\OtherTok{=} \FunctionTok{rnorm}\NormalTok{(}\AttributeTok{n =}\NormalTok{ size, }\AttributeTok{mean =} \DecValTok{20000}\NormalTok{, }\AttributeTok{sd =} \DecValTok{7000}\NormalTok{)}
\NormalTok{y\_1 }\OtherTok{=} \FunctionTok{rnorm}\NormalTok{(}\AttributeTok{n =}\NormalTok{ size, }\AttributeTok{mean =} \DecValTok{21500}\NormalTok{, }\AttributeTok{sd =} \DecValTok{8000}\NormalTok{)}
\NormalTok{decision }\OtherTok{=} \FunctionTok{ifelse}\NormalTok{(y\_1 }\SpecialCharTok{\textgreater{}}\NormalTok{ y\_0 }\SpecialCharTok{{-}} \DecValTok{1000}\NormalTok{, }\DecValTok{1}\NormalTok{, }\DecValTok{0}\NormalTok{)}

\NormalTok{df }\OtherTok{=} \FunctionTok{data.frame}\NormalTok{(y\_0, y\_1, decision)}
\end{Highlighting}
\end{Shaded}

\begin{enumerate}
\def\labelenumi{\arabic{enumi}.}
\tightlist
\item
  What is the average treatment effect in your sample? How does it
  compare to the true average treatment effect?
\end{enumerate}

True Average Treatment Effect is 1500

Average Treatment Effect:

\begin{Shaded}
\begin{Highlighting}[]
\NormalTok{sample\_ate }\OtherTok{=} \FunctionTok{mean}\NormalTok{(df[df}\SpecialCharTok{$}\NormalTok{decision }\SpecialCharTok{==} \DecValTok{1}\NormalTok{, }\StringTok{"y\_1"}\NormalTok{]) }\SpecialCharTok{{-}} \FunctionTok{mean}\NormalTok{(df[df}\SpecialCharTok{$}\NormalTok{decision }\SpecialCharTok{==} \DecValTok{0}\NormalTok{, }\StringTok{"y\_0"}\NormalTok{])}
\NormalTok{sample\_ate}
\end{Highlighting}
\end{Shaded}

\begin{verbatim}
## [1] 1019.282
\end{verbatim}

Difference between True Average Treatment Effect and Sample Average
Treatment Effect

\begin{Shaded}
\begin{Highlighting}[]
\DecValTok{1500} \SpecialCharTok{{-}}\NormalTok{ sample\_ate}
\end{Highlighting}
\end{Shaded}

\begin{verbatim}
## [1] 480.7184
\end{verbatim}

FINISH

\begin{enumerate}
\def\labelenumi{\arabic{enumi}.}
\setcounter{enumi}{1}
\tightlist
\item
  What is the distribution of compliers, always takers, and never takers
  in your sample (i.e.~what is P(A), P(C), and P(N))?
\end{enumerate}

\begin{Shaded}
\begin{Highlighting}[]
\CommentTok{\#Always Takers}
\CommentTok{\#Always goes to college regardless of treatment status}
\CommentTok{\#Always Takers will go to college if in treatment and Y\_1 \textgreater{} Y\_0}
\CommentTok{\#Always Takers will go to college if not in treatment and Y\_1 {-} 1000 \textgreater{} Y\_0}
\CommentTok{\#If Y\_1 {-} Y\_0 \textgreater{} 1000 and Y\_1 {-} Y\_0 \textgreater{} 0, then individuals are always takers}
\CommentTok{\#Thus if Y\_1 {-} Y\_0 \textgreater{} 1000, individuals are always takers}
\NormalTok{percent }\OtherTok{=}\NormalTok{ df }\SpecialCharTok{\%\textgreater{}\%}
  \FunctionTok{filter}\NormalTok{(y\_1 }\SpecialCharTok{{-}}\NormalTok{ y\_0 }\SpecialCharTok{\textgreater{}} \DecValTok{1000}\NormalTok{) }\SpecialCharTok{\%\textgreater{}\%}
  \FunctionTok{summarize}\NormalTok{(}\AttributeTok{always\_takers =} \FunctionTok{n}\NormalTok{()}\SpecialCharTok{/}\NormalTok{size)}


\CommentTok{\#Never Takers}
\CommentTok{\#Never goes to college regardless of treatment status}
\CommentTok{\#Never Takers will not go to college if in treatment and Y\_1 \textless{} Y\_0}
\CommentTok{\#Never Takers will not go to college if not in treatment and Y\_1 {-} 1000 \textless{} Y\_0}
\CommentTok{\#If Y\_1 {-} Y\_0 \textless{} 1000 \& Y\_1 {-} Y\_0 \textless{} 0, then individuals are never takers}
\CommentTok{\#Thus if Y\_1 {-} Y\_0 \textless{} 0, individuals are never takers}
\NormalTok{percent[}\DecValTok{1}\NormalTok{,}\DecValTok{2}\NormalTok{] }\OtherTok{=}\NormalTok{ df }\SpecialCharTok{\%\textgreater{}\%}
  \FunctionTok{filter}\NormalTok{(y\_1 }\SpecialCharTok{{-}}\NormalTok{ y\_0 }\SpecialCharTok{\textless{}} \DecValTok{0}\NormalTok{) }\SpecialCharTok{\%\textgreater{}\%}
  \FunctionTok{summarize}\NormalTok{(}\AttributeTok{never\_takers =} \FunctionTok{n}\NormalTok{()}\SpecialCharTok{/}\NormalTok{size)}

\CommentTok{\#Compliers}
\CommentTok{\#Complies with Treatment}
\CommentTok{\#Compliers go to college if in treatment and Y\_1 \textgreater{} Y\_0}
\CommentTok{\#Compliers do not go to college in not in treatment and Y\_1 {-} 1000 \textless{} Y\_0}
\CommentTok{\#If 0 \textless{} Y\_1 {-} Y\_0 \textless{} 1000, then individuals are compliers}
\NormalTok{percent[}\DecValTok{1}\NormalTok{,}\DecValTok{3}\NormalTok{] }\OtherTok{=}\NormalTok{ df }\SpecialCharTok{\%\textgreater{}\%}
  \FunctionTok{filter}\NormalTok{(y\_1 }\SpecialCharTok{{-}}\NormalTok{ y\_0 }\SpecialCharTok{\textless{}=} \DecValTok{1000}\NormalTok{) }\SpecialCharTok{\%\textgreater{}\%}
  \FunctionTok{filter}\NormalTok{(y\_1 }\SpecialCharTok{{-}}\NormalTok{ y\_0 }\SpecialCharTok{\textgreater{}=} \DecValTok{0}\NormalTok{) }\SpecialCharTok{\%\textgreater{}\%}
  \FunctionTok{summarize}\NormalTok{(}\AttributeTok{compliers =} \FunctionTok{n}\NormalTok{()}\SpecialCharTok{/}\NormalTok{size)}

\NormalTok{percent}\SpecialCharTok{$}\NormalTok{sum }\OtherTok{=} \FunctionTok{sum}\NormalTok{(percent[}\DecValTok{1}\NormalTok{,])}

\NormalTok{percent}
\end{Highlighting}
\end{Shaded}

\begin{verbatim}
##   always_takers never_takers compliers sum
## 1        0.5206       0.4418    0.0376   1
\end{verbatim}

\begin{enumerate}
\def\labelenumi{\arabic{enumi}.}
\setcounter{enumi}{2}
\tightlist
\item
  What is the average impact of the training for compliers, always
  takers, and never takers in your sample?
\end{enumerate}

CHECK THIS PROBLEM

\begin{Shaded}
\begin{Highlighting}[]
\CommentTok{\#Average Impact of Training}

\NormalTok{impact }\OtherTok{=}\NormalTok{ df }\SpecialCharTok{\%\textgreater{}\%}
  \FunctionTok{filter}\NormalTok{(y\_1 }\SpecialCharTok{{-}}\NormalTok{ y\_0 }\SpecialCharTok{\textgreater{}} \DecValTok{1000}\NormalTok{) }\SpecialCharTok{\%\textgreater{}\%}
  \FunctionTok{summarize}\NormalTok{(}\AttributeTok{always\_takers =} \FunctionTok{mean}\NormalTok{(y\_1) }\SpecialCharTok{{-}} \FunctionTok{mean}\NormalTok{(y\_0))}

\NormalTok{impact[}\DecValTok{1}\NormalTok{,}\DecValTok{2}\NormalTok{] }\OtherTok{=}\NormalTok{ df }\SpecialCharTok{\%\textgreater{}\%}
  \FunctionTok{filter}\NormalTok{(y\_1 }\SpecialCharTok{{-}}\NormalTok{ y\_0 }\SpecialCharTok{\textless{}} \DecValTok{0}\NormalTok{) }\SpecialCharTok{\%\textgreater{}\%}
  \FunctionTok{summarize}\NormalTok{(}\AttributeTok{never\_takers =} \FunctionTok{mean}\NormalTok{(y\_1) }\SpecialCharTok{{-}} \FunctionTok{mean}\NormalTok{(y\_0))}

\NormalTok{impact[}\DecValTok{1}\NormalTok{,}\DecValTok{3}\NormalTok{] }\OtherTok{=}\NormalTok{ df }\SpecialCharTok{\%\textgreater{}\%}
  \FunctionTok{filter}\NormalTok{(y\_1 }\SpecialCharTok{{-}}\NormalTok{ y\_0 }\SpecialCharTok{\textless{}=} \DecValTok{1000}\NormalTok{) }\SpecialCharTok{\%\textgreater{}\%}
  \FunctionTok{filter}\NormalTok{(y\_1 }\SpecialCharTok{{-}}\NormalTok{ y\_0 }\SpecialCharTok{\textgreater{}=} \DecValTok{0}\NormalTok{) }\SpecialCharTok{\%\textgreater{}\%}
  \FunctionTok{summarize}\NormalTok{(}\AttributeTok{compliers =} \FunctionTok{mean}\NormalTok{(y\_1) }\SpecialCharTok{{-}} \FunctionTok{mean}\NormalTok{(y\_0))}

\NormalTok{impact}
\end{Highlighting}
\end{Shaded}

\begin{verbatim}
##   always_takers never_takers compliers
## 1      9669.758    -7925.623  518.2911
\end{verbatim}

\begin{Shaded}
\begin{Highlighting}[]
\CommentTok{\#impact[1,1] * percent[1,1] + impact[1,2] * percent[1,2] + impact[1,3] * percent[1,3]}
\end{Highlighting}
\end{Shaded}

\begin{enumerate}
\def\labelenumi{\arabic{enumi}.}
\setcounter{enumi}{3}
\tightlist
\item
  Why is it reasonable to assume there are no defiers given our
  assumptions about how people are making participation decisions?
\end{enumerate}

It is reasonable to assume there are no defiers given our assumption
because individuals are either better off with treatment
(always-takers), without treatment (never-takers), or with treatment if
provided (compliers). Specifically to be a defier, it must be the case
that:

Defiers go to college if not in treatment which implies
\(Y_1 - 1000 > Y_0\)

Defiers choose to not go to college if in treatment which implies
\(Y_1 < Y_0\)

Thus to be a defier, it must be that \$ 0 \textgreater{} Y\_1 - Y\_0
\textgreater{} 1000\$ which is impossible.

\newpage

So far, we have been using the full sample because we observe both
potential outcomes. Now, let's pretend we are in the real world and only
observe the outcome that results from someone's participation decision.
To this end, randomly assign half of your sample to a treatment group
and half to a control group. Generate an indicator P that equals 1 if
someone receives the training and 0 otherwise. Remember: we have assumed
people make participation decisions entirely based on their earnings
with or without the training less any costs of the training. This should
depend on the observations treatment status.

Generate a variable Y equal to observed earnings using the following
formula:

\[
Y = PY_1 + (1-P)Y_0
\]

\begin{Shaded}
\begin{Highlighting}[]
\NormalTok{df}\SpecialCharTok{$}\NormalTok{treatment }\OtherTok{=} \FunctionTok{rbinom}\NormalTok{(}\AttributeTok{n =}\NormalTok{ size, }\AttributeTok{size =} \DecValTok{1}\NormalTok{, }\AttributeTok{prob =} \FloatTok{0.5}\NormalTok{)}
\NormalTok{df }\OtherTok{=}\NormalTok{ df }\SpecialCharTok{\%\textgreater{}\%}
  \FunctionTok{mutate}\NormalTok{(}\AttributeTok{y =}\NormalTok{ decision}\SpecialCharTok{*}\NormalTok{y\_1 }\SpecialCharTok{+}\NormalTok{ (}\DecValTok{1}\SpecialCharTok{{-}}\NormalTok{decision)}\SpecialCharTok{*}\NormalTok{y\_0)}
\end{Highlighting}
\end{Shaded}

\begin{enumerate}
\def\labelenumi{\arabic{enumi}.}
\setcounter{enumi}{4}
\tightlist
\item
  Use a regression to estimate the intent-to-treat effect in your
  sample. What is the point estimate and the 95\% confidence interval
  around the estimate?
\end{enumerate}

\begin{Shaded}
\begin{Highlighting}[]
\NormalTok{sum1 }\OtherTok{=} \FunctionTok{summary}\NormalTok{(}\FunctionTok{lm}\NormalTok{(}\AttributeTok{data =}\NormalTok{ df, y }\SpecialCharTok{\textasciitilde{}}\NormalTok{ treatment))}

\NormalTok{a }\OtherTok{=}\NormalTok{ sum1}\SpecialCharTok{$}\NormalTok{coefficients[}\DecValTok{2}\NormalTok{,}\DecValTok{1}\NormalTok{]}
\NormalTok{b }\OtherTok{=}\NormalTok{ sum1}\SpecialCharTok{$}\NormalTok{coefficients[}\DecValTok{2}\NormalTok{,}\DecValTok{1}\NormalTok{] }\SpecialCharTok{+} \DecValTok{2}\SpecialCharTok{*}\NormalTok{sum1}\SpecialCharTok{$}\NormalTok{coefficients[}\DecValTok{2}\NormalTok{,}\DecValTok{2}\NormalTok{]}
\NormalTok{c }\OtherTok{=}\NormalTok{ sum1}\SpecialCharTok{$}\NormalTok{coefficients[}\DecValTok{2}\NormalTok{,}\DecValTok{1}\NormalTok{] }\SpecialCharTok{{-}} \DecValTok{2}\SpecialCharTok{*}\NormalTok{sum1}\SpecialCharTok{$}\NormalTok{coefficients[}\DecValTok{2}\NormalTok{,}\DecValTok{1}\NormalTok{]}

\NormalTok{table }\OtherTok{=} \FunctionTok{cbind}\NormalTok{(a,b,c)}
\FunctionTok{colnames}\NormalTok{(table) }\OtherTok{=} \FunctionTok{c}\NormalTok{(}\StringTok{"Lower Bound"}\NormalTok{, }\StringTok{"Estimate"}\NormalTok{, }\StringTok{"Upper Bound"}\NormalTok{)}
\NormalTok{table}
\end{Highlighting}
\end{Shaded}

\begin{verbatim}
##      Lower Bound Estimate Upper Bound
## [1,]   -4.530482  248.117    4.530482
\end{verbatim}

\begin{enumerate}
\def\labelenumi{\arabic{enumi}.}
\setcounter{enumi}{5}
\tightlist
\item
  Use two-stage least squares to estimate the local average treatment
  effect in your sample. Comment on the point estimate and the 95\%
  confidence interval around the estimate. How does this compare to the
  effects we estimated earlier in this problem?
\end{enumerate}

ASK ABOUT!

\begin{Shaded}
\begin{Highlighting}[]
\FunctionTok{p\_load}\NormalTok{(fixest)}

\CommentTok{\#First Stage}
\FunctionTok{feols}\NormalTok{(}\AttributeTok{data =}\NormalTok{ df, y }\SpecialCharTok{\textasciitilde{}}\NormalTok{ treatment }\SpecialCharTok{|}\NormalTok{ decision)}
\end{Highlighting}
\end{Shaded}

\begin{verbatim}
## OLS estimation, Dep. Var.: y
## Observations: 10,000 
## Fixed-effects: decision: 2
## Standard-errors: Clustered (decision) 
##           Estimate Std. Error t value Pr(>|t|) 
## treatment  -5.3592      3.525 -1.5204 0.370382 
## ---
## Signif. codes:  0 '***' 0.001 '**' 0.01 '*' 0.05 '.' 0.1 ' ' 1
## RMSE: 6,295.6     Adj. R2: 0.006084
##                 Within R2: 1.812e-7
\end{verbatim}

\begin{Shaded}
\begin{Highlighting}[]
\CommentTok{\#Second Stage}
\FunctionTok{summary}\NormalTok{(}\FunctionTok{lm}\NormalTok{(}\AttributeTok{data =}\NormalTok{ df, y }\SpecialCharTok{\textasciitilde{}}\NormalTok{ treatment))}
\end{Highlighting}
\end{Shaded}

\begin{verbatim}
## 
## Call:
## lm(formula = y ~ treatment, data = df)
## 
## Residuals:
##      Min       1Q   Median       3Q      Max 
## -26262.2  -4315.4   -306.8   4109.5  27430.1 
## 
## Coefficients:
##             Estimate Std. Error t value Pr(>|t|)    
## (Intercept) 25059.48      89.08 281.305   <2e-16 ***
## treatment      -4.53     126.32  -0.036    0.971    
## ---
## Signif. codes:  0 '***' 0.001 '**' 0.01 '*' 0.05 '.' 0.1 ' ' 1
## 
## Residual standard error: 6316 on 9998 degrees of freedom
## Multiple R-squared:  1.286e-07,  Adjusted R-squared:  -9.989e-05 
## F-statistic: 0.001286 on 1 and 9998 DF,  p-value: 0.9714
\end{verbatim}

\begin{Shaded}
\begin{Highlighting}[]
\CommentTok{\#Reduced Form}
\end{Highlighting}
\end{Shaded}

\begin{enumerate}
\def\labelenumi{\arabic{enumi}.}
\setcounter{enumi}{6}
\tightlist
\item
  Re-run your code but drawing a sample of 1,000,000 observations
  instead of 10,000. How does the estimated LATE compare to the earlier
  treatment effects now?
\end{enumerate}

FIGURE OUT

\begin{Shaded}
\begin{Highlighting}[]
\NormalTok{size }\OtherTok{=} \DecValTok{1000000}
\end{Highlighting}
\end{Shaded}

\newpage

\hypertarget{problem-2}{%
\section{Problem 2}\label{problem-2}}

Suppose a researcher is evaluating an experiment using a sample that
consists of 50\% compliers, 25\% always takers, and 25\% never takers.
The researcher decides to estimate the intent-to-treat effect but
dropping people who did not comply with the treatment protocol from the
sample. After dropping people who did not comply with their treatment
assignment from the sample, what is the distribution of compliers,
always takers, and never takers in the treatment group? In the control
group? Why is this a problem?

Solution:

There would be no never takers in the treatment group (since they
wouldn't comply with treatment), all of the always takers who were
selected for treatment and all of the compliers who were selected for
treatment. In the control group, there would be no always takers, some
of the never takers who were selected for control and all of the
compliers who were selected for treatment.

This is a problem because we are still including the effect of never
takers and always takers in our treatment effect. What we truly want to
know is the effect of treatment for specifically those who complied.
Never takers and always takers may skew our results and/or make our
results not as generalizable.

CHECK

\newpage

\hypertarget{problem-3}{%
\section{Problem 3}\label{problem-3}}

Table 1 below includes details about the total sample size, probability
of treatment, and treatment effect within each of 5 blocks. What would
the pooled treatment effect be if estimated using an OLS regression of
the outcome on treatment and block fixed effects? How much weight does
each block get in the pooled estimate?

CHECK

\begin{Shaded}
\begin{Highlighting}[]
\NormalTok{block }\OtherTok{=} \FunctionTok{seq}\NormalTok{(}\DecValTok{1}\SpecialCharTok{:}\DecValTok{5}\NormalTok{)}
\NormalTok{table }\OtherTok{=} \FunctionTok{data.frame}\NormalTok{(block)}
\NormalTok{table}\SpecialCharTok{$}\NormalTok{N }\OtherTok{=} \FunctionTok{c}\NormalTok{(}\DecValTok{100}\NormalTok{, }\DecValTok{100}\NormalTok{, }\DecValTok{200}\NormalTok{, }\DecValTok{200}\NormalTok{, }\DecValTok{300}\NormalTok{)}
\NormalTok{table}\SpecialCharTok{$}\NormalTok{prob\_T }\OtherTok{=} \FunctionTok{c}\NormalTok{(}\FloatTok{0.5}\NormalTok{, }\FloatTok{0.25}\NormalTok{, }\FloatTok{0.5}\NormalTok{, }\FloatTok{0.75}\NormalTok{, }\FloatTok{0.1}\NormalTok{)}
\NormalTok{table}\SpecialCharTok{$}\NormalTok{treatment\_effect }\OtherTok{=} \FunctionTok{c}\NormalTok{(}\SpecialCharTok{{-}}\DecValTok{1}\NormalTok{, }\DecValTok{0}\NormalTok{, }\DecValTok{1}\NormalTok{, }\DecValTok{2}\NormalTok{, }\DecValTok{3}\NormalTok{)}

\NormalTok{table}
\end{Highlighting}
\end{Shaded}

\begin{verbatim}
##   block   N prob_T treatment_effect
## 1     1 100   0.50               -1
## 2     2 100   0.25                0
## 3     3 200   0.50                1
## 4     4 200   0.75                2
## 5     5 300   0.10                3
\end{verbatim}

\begin{Shaded}
\begin{Highlighting}[]
\CommentTok{\#Each blocks variance formula}
\NormalTok{table}\SpecialCharTok{$}\NormalTok{block\_variance }\OtherTok{=}\NormalTok{ table}\SpecialCharTok{$}\NormalTok{prob\_T}\SpecialCharTok{*}\NormalTok{(}\DecValTok{1}\SpecialCharTok{{-}}\NormalTok{table}\SpecialCharTok{$}\NormalTok{prob\_T)}\SpecialCharTok{*}\NormalTok{table}\SpecialCharTok{$}\NormalTok{N}
\CommentTok{\#Number of treated individuals in each block}
\NormalTok{table}\SpecialCharTok{$}\NormalTok{treated }\OtherTok{=}\NormalTok{ table}\SpecialCharTok{$}\NormalTok{N}\SpecialCharTok{*}\NormalTok{table}\SpecialCharTok{$}\NormalTok{prob\_T}
\CommentTok{\#Total treatment probability}
\NormalTok{treatedprob }\OtherTok{=} \FunctionTok{sum}\NormalTok{(table}\SpecialCharTok{$}\NormalTok{treated)}\SpecialCharTok{/}\FunctionTok{sum}\NormalTok{(table}\SpecialCharTok{$}\NormalTok{N)}
\CommentTok{\#Total variance}
\NormalTok{total\_var }\OtherTok{=}\NormalTok{ treatedprob}\SpecialCharTok{*}\NormalTok{(}\DecValTok{1}\SpecialCharTok{{-}}\NormalTok{treatedprob)}\SpecialCharTok{*}\FunctionTok{sum}\NormalTok{(table}\SpecialCharTok{$}\NormalTok{N)}
\CommentTok{\#Weights for each block}
\NormalTok{table}\SpecialCharTok{$}\NormalTok{weights }\OtherTok{=}\NormalTok{ table}\SpecialCharTok{$}\NormalTok{N}\SpecialCharTok{/}\FunctionTok{sum}\NormalTok{(table}\SpecialCharTok{$}\NormalTok{N)}\SpecialCharTok{*}\NormalTok{table}\SpecialCharTok{$}\NormalTok{block\_variance}\SpecialCharTok{/}\NormalTok{total\_var}


\FunctionTok{sum}\NormalTok{(table}\SpecialCharTok{$}\NormalTok{N}\SpecialCharTok{/}\FunctionTok{sum}\NormalTok{(table}\SpecialCharTok{$}\NormalTok{N))}
\end{Highlighting}
\end{Shaded}

\begin{verbatim}
## [1] 1
\end{verbatim}

\begin{Shaded}
\begin{Highlighting}[]
\FunctionTok{sum}\NormalTok{(table}\SpecialCharTok{$}\NormalTok{block\_variance}\SpecialCharTok{/}\NormalTok{total\_var)}
\end{Highlighting}
\end{Shaded}

\begin{verbatim}
## [1] 0.7361416
\end{verbatim}

\begin{Shaded}
\begin{Highlighting}[]
\CommentTok{\#Pooled Treatment Effect}
\FunctionTok{sum}\NormalTok{(table}\SpecialCharTok{$}\NormalTok{weights}\SpecialCharTok{*}\NormalTok{table}\SpecialCharTok{$}\NormalTok{treatment\_effect)}
\end{Highlighting}
\end{Shaded}

\begin{verbatim}
## [1] 0.2418917
\end{verbatim}

Note for future reference:

Pooled Treatment Effect =
\(\sum_b \dfrac{n_b}{N}*\dfrac{Var(T_{i,b})}{Var(T_i)}*\hat{\Delta}_b\).

Let \(\sum_b n_b = N\). Assume \(\hat{\Delta}_b = c, \forall b\)

\$\$ \sum\_b
\dfrac{n_b}{N}\emph{\dfrac{Var(T_{i,b})}{Var(T_i)}}\hat{\Delta}\_b =
c*\sum\_b \dfrac{n_b}{N}*\dfrac{Var(T_{i,b})}{Var(T_i)} \textbackslash{}

= c*\sum\_b \dfrac{n_b}{N}*\dfrac{n_b*p_b*(1-p_b)}{N*p*(1-p)} \$\$
Assume \(p = 0.5\) and \(p_b = 0.5, \forall b\)

\[
c*\sum_b \dfrac{n_b}{N}*\dfrac{n_b*p_b*(1-p_b)}{N*p*(1-p)} = c*\sum_b \dfrac{n_b^2}{N^2} \\
\]

\newpage

\hypertarget{problem-4}{%
\section{Problem 4}\label{problem-4}}

In class, we saw how Moulton's design effect can be used to approximate
the impact of clustering on our standard error. This problem will give
you practice using this adjustment.

Imagine you are considering running a cluster randomized experiment with
10,000 observations split evenly across 118 clusters. You will randomly
assign half of clusters to treatment and control. Your main outcome of
interest is a binary variable that historically has equaled one for 72.9
percent of observations.

\begin{enumerate}
\def\labelenumi{\arabic{enumi}.}
\item
  What is your minimum detectable effect if the intracluster correlation
  is 0?
\item
  What is your minimum detectable effect if the intracluster correlation
  is 0.2?
\item
  How do these answers change if you can control for baseline
  characteristics that explain 20 percent of residual variation?
\end{enumerate}

\newpage

\hypertarget{problem-5}{%
\section{Problem 5}\label{problem-5}}

We often focus on identifying and estimating average treatment effects
like:

\[
E[Y | T = 1] - E[Y | T = 0]E[Y | T = 1] - E[Y | T = 0]
\]

Discuss why this rule implies that the variance of treatment effects is
not identified by randomization into treatment or control alone.

Propose an additional assumption that would allow you to identify the
variance of treatment effects. Is it plausible?

\end{document}
